% arara: xelatex
\documentclass[12pt]{article}

% \usepackage{physics}

\usepackage{hyperref}
\hypersetup{
    colorlinks=true,
    linkcolor=blue,
    filecolor=magenta,      
    urlcolor=cyan,
    pdftitle={Overleaf Example},
    pdfpagemode=FullScreen,
    }

\usepackage{tikzducks}

\usepackage{tikz} % картинки в tikz
\usetikzlibrary{shapes, arrows, positioning}
\usepackage{microtype} % свешивание пунктуации

\usepackage{array} % для столбцов фиксированной ширины

\usepackage{indentfirst} % отступ в первом параграфе

\usepackage{sectsty} % для центрирования названий частей
\allsectionsfont{\centering}

\usepackage{amsmath, amsfonts, amssymb} % куча стандартных математических плюшек

\usepackage{comment}

\usepackage[top=2cm, left=1.2cm, right=1.2cm, bottom=2cm]{geometry} % размер текста на странице

\usepackage{lastpage} % чтобы узнать номер последней страницы

\usepackage{enumitem} % дополнительные плюшки для списков
%  например \begin{enumerate}[resume] позволяет продолжить нумерацию в новом списке
\usepackage{caption}

\usepackage{url} % to use \url{link to web}


\newcommand{\smallduck}{\begin{tikzpicture}[scale=0.3]
    \duck[
        cape=black,
        hat=black,
        mask=black
    ]
    \end{tikzpicture}}

\usepackage{fancyhdr} % весёлые колонтитулы
\pagestyle{fancy}
\lhead{Time Series and Stochastic Processes}
\chead{}
\rhead{Warm Winter Exam Retake, 2025-03-24}
\lfoot{}
\cfoot{}
\rfoot{}

\renewcommand{\headrulewidth}{0.4pt}
\renewcommand{\footrulewidth}{0.4pt}

\usepackage{tcolorbox} % рамочки!

\usepackage{todonotes} % для вставки в документ заметок о том, что осталось сделать
% \todo{Здесь надо коэффициенты исправить}
% \missingfigure{Здесь будет Последний день Помпеи}
% \listoftodos - печатает все поставленные \todo'шки


% более красивые таблицы
\usepackage{booktabs}
% заповеди из докупентации:
% 1. Не используйте вертикальные линни
% 2. Не используйте двойные линии
% 3. Единицы измерения - в шапку таблицы
% 4. Не сокращайте .1 вместо 0.1
% 5. Повторяющееся значение повторяйте, а не говорите "то же"


\setcounter{MaxMatrixCols}{20}
% by crazy default pmatrix supports only 10 cols :)


\usepackage{fontspec}
\usepackage{libertine}
\usepackage{polyglossia}

\setmainlanguage{russian}
\setotherlanguages{english}

% download "Linux Libertine" fonts:
% http://www.linuxlibertine.org/index.php?id=91&L=1
% \setmainfont{Linux Libertine O} % or Helvetica, Arial, Cambria
% why do we need \newfontfamily:
% http://tex.stackexchange.com/questions/91507/
% \newfontfamily{\cyrillicfonttt}{Linux Libertine O}

\AddEnumerateCounter{\asbuk}{\russian@alph}{щ} % для списков с русскими буквами
% \setlist[enumerate, 2]{label=\asbuk*),ref=\asbuk*}

%% эконометрические сокращения
\DeclareMathOperator{\Cov}{\mathbb{C}ov}
\DeclareMathOperator{\Corr}{\mathbb{C}orr}
\DeclareMathOperator{\pCorr}{pCorr}
\DeclareMathOperator{\Var}{\mathbb{V}ar}
\DeclareMathOperator{\col}{col}
\DeclareMathOperator{\row}{row}

\let\P\relax
\DeclareMathOperator{\P}{\mathbb{P}}

\DeclareMathOperator{\E}{\mathbb{E}}
% \DeclareMathOperator{\tr}{trace}
\DeclareMathOperator{\card}{card}

\DeclareMathOperator{\Convex}{Convex}
\DeclareMathOperator{\plim}{plim}

\newcommand{\cN}{\mathcal{N}}
\newcommand{\cF}{\mathcal{F}}

\newcommand{\RR}{\mathbb{R}}
\newcommand{\NN}{\mathbb{N}}
\newcommand{\hb}{\hat{\beta}}
\newcommand{\dPois}{\mathrm{Pois}}


\newcommand{\dExpo}{\mathrm{Expo}}




\begin{document}

\begin{enumerate}
    \item {[10]} The process $(u_t)$ is a white noise with $\Var(u_t) = \sigma^2$.
    Consider the process 
    \[
    y_t = (2 + 3t) u_1 + (3 + 2t) u_t.
    \]
    \begin{enumerate}
        \item {[4]} Find $\E(y_t)$ and $\Var(y_t)$.
        \item {[4]} Find $\Cov(y_t, y_s)$.
        \item {[2]} Is the process $(y_t)$ stationary?
    \end{enumerate}

    \item {[10]} Consider the stationary solution of the equation 
    $y_t = 5 + 0.7 y_{t-1} + u_t - 0.5 u_{t-1}$,    
    where $(u_t)$ is a white noise process with variance $40$. 
    \begin{enumerate}
        \item {[4]} If possible rewrite this solution as $AR(\infty)$ process. 
        \item {[4]} If possible rewrite this solution as $MA(\infty)$ process. 
        \item {[2]} Find $\Cov(u_t, \Delta y_s)$ for this solution. 
    \end{enumerate}
    
    \item {[10]} Consider the equation 
    $y_t = 6 + 0.4 y_{t-1} - 0.12 y_{t-2} + u_t + 0.7u_{t-1}$,
    where $(u_t)$ is a white noise. 

    \begin{enumerate}
        \item {[1]} How many non-stationary solutions does this equation have?
        \item {[4]} How many stationary solutions of $MA(\infty)$ form with respect to $(u_t)$ does this equation have?
        \item {[3]} Can we rewrite the stationary solution in $AR(\infty)$ form with respect to $(u_t)$?
        \item {[2]} Find $\E(y_t)$ for the stationary solution.
        % \item Write a different equation that has a unique stationary solution with the same autocovariance function as the stationary solution of the given equation.
    \end{enumerate}


    \item {[10]} Let $(y_t)$ be the solution of the equation 
    $y_t = 3t + 2 y_{t-1} - y_{t-2} + u_t$,
    where $(u_t)$ are independent and normally distributed $\cN(0;9)$ and $y_0$ is a constant.

    \begin{enumerate}
        \item {[5]} Find 95\% confidence for $y_{101}$ given that $y_{100} = 3$ and $y_{99} = 4$.
        \item {[5]} Find 95\% confidence for $y_{102}$ given that $y_{100} = 3$ and $y_{99} = 4$.
    \end{enumerate}
    
    \item {[10]} For the stationary solution of the equation $y_t = 5 + 0.4 y_{t-1} - 0.01 y_{t-2} + u_t$, 
    where $(u_t)$ is a white noise process. 
    \begin{enumerate}
        \item {[5]} Find the first three values of the autocorrelation function $\rho_1$, $\rho_2$, $\rho_3$.
        \item {[5]} Find all values of the partial autocorrelation function $\phi_{kk}$.
    \end{enumerate}

    \item {[10]} Let $(y_t)$ be a stationary solution of $(1+0.5L)y_t = (1 + 0.5F)u_t$,
    where $L$ is the lag operator, $F$ is the forward operator and $(u_t)$ is a white noise with $\Var(u_t) = 1$.
    \begin{enumerate}
        \item {[5]} Find $\alpha$ and $\beta$ in the representation $y_t = \alpha y_{t+1} + \beta y_t + \dots$.
        \item {[5]} Find the autocovariance function of $(u_t)$.
    \end{enumerate}
\end{enumerate}


\end{document}


% arara: xelatex
\documentclass[12pt]{article}

% \usepackage{physics}

\usepackage{hyperref}
\hypersetup{
    colorlinks=true,
    linkcolor=blue,
    filecolor=magenta,      
    urlcolor=cyan,
    pdftitle={Overleaf Example},
    pdfpagemode=FullScreen,
    }

\usepackage{tikzducks}

\usepackage{tikz} % картинки в tikz
\usetikzlibrary{shapes, arrows, positioning}
\usepackage{microtype} % свешивание пунктуации

\usepackage{array} % для столбцов фиксированной ширины

\usepackage{indentfirst} % отступ в первом параграфе

\usepackage{sectsty} % для центрирования названий частей
\allsectionsfont{\centering}

\usepackage{amsmath, amsfonts, amssymb} % куча стандартных математических плюшек

\usepackage{comment}

\usepackage[top=2cm, left=1.2cm, right=1.2cm, bottom=2cm]{geometry} % размер текста на странице

\usepackage{lastpage} % чтобы узнать номер последней страницы

\usepackage{enumitem} % дополнительные плюшки для списков
%  например \begin{enumerate}[resume] позволяет продолжить нумерацию в новом списке
\usepackage{caption}

\usepackage{url} % to use \url{link to web}


\newcommand{\smallduck}{\begin{tikzpicture}[scale=0.3]
    \duck[
        cape=black,
        hat=black,
        mask=black
    ]
    \end{tikzpicture}}

\usepackage{fancyhdr} % весёлые колонтитулы
\pagestyle{fancy}
\lhead{Time Series and Stochastic Processes}
\chead{}
\rhead{Midterm exam-2}
\lfoot{}
\cfoot{}
\rfoot{}

\renewcommand{\headrulewidth}{0.4pt}
\renewcommand{\footrulewidth}{0.4pt}

\usepackage{tcolorbox} % рамочки!

\usepackage{todonotes} % для вставки в документ заметок о том, что осталось сделать
% \todo{Здесь надо коэффициенты исправить}
% \missingfigure{Здесь будет Последний день Помпеи}
% \listoftodos - печатает все поставленные \todo'шки


% более красивые таблицы
\usepackage{booktabs}
% заповеди из докупентации:
% 1. Не используйте вертикальные линни
% 2. Не используйте двойные линии
% 3. Единицы измерения - в шапку таблицы
% 4. Не сокращайте .1 вместо 0.1
% 5. Повторяющееся значение повторяйте, а не говорите "то же"


\setcounter{MaxMatrixCols}{20}
% by crazy default pmatrix supports only 10 cols :)


\usepackage{fontspec}
\usepackage{libertine}
\usepackage{polyglossia}

\setmainlanguage{russian}
\setotherlanguages{english}

% download "Linux Libertine" fonts:
% http://www.linuxlibertine.org/index.php?id=91&L=1
% \setmainfont{Linux Libertine O} % or Helvetica, Arial, Cambria
% why do we need \newfontfamily:
% http://tex.stackexchange.com/questions/91507/
% \newfontfamily{\cyrillicfonttt}{Linux Libertine O}

\AddEnumerateCounter{\asbuk}{\russian@alph}{щ} % для списков с русскими буквами
% \setlist[enumerate, 2]{label=\asbuk*),ref=\asbuk*}

%% эконометрические сокращения
\DeclareMathOperator{\Cov}{\mathbb{C}ov}
\DeclareMathOperator{\Corr}{\mathbb{C}orr}
\DeclareMathOperator{\Var}{\mathbb{V}ar}
\DeclareMathOperator{\col}{col}
\DeclareMathOperator{\row}{row}

\let\P\relax
\DeclareMathOperator{\P}{\mathbb{P}}

\DeclareMathOperator{\E}{\mathbb{E}}
% \DeclareMathOperator{\tr}{trace}
\DeclareMathOperator{\card}{card}

\DeclareMathOperator{\Convex}{Convex}
\DeclareMathOperator{\plim}{plim}

\newcommand{\cN}{\mathcal{N}}
\newcommand{\cF}{\mathcal{F}}

\newcommand{\RR}{\mathbb{R}}
\newcommand{\NN}{\mathbb{N}}
\newcommand{\hb}{\hat{\beta}}
\newcommand{\dPois}{\mathrm{Pois}}


\newcommand{\dExpo}{\mathrm{Expo}}




\begin{document}

\begin{enumerate}
    \item {[10]} I have two pockets: left and right. 
    I also have two indistinguishable coins. 
    Initially they are both in the left pocket. 
    Each moment of time I randomly select one of my pockets. 
    If it is empty I do nothing. 
    It it is not empty I move a coin from the selected pocket to another one. 

    Consider the Markov chain where the state is the location of the two coins. 
    \begin{enumerate}
        \item {[5]} Draw the diagram and find the transition matrix.
        \item {[2]} Classify the states.
        \item {[3]} Which proportion of my eternal life the coins are split in both pockets?
    \end{enumerate}


    \item {[10]} The random variable $X$, $Y$ and $Z$ are independent and normally distributed. 
    Consider the sigma-algebras
    \[
        \cF_1 = \sigma(X, Y), \quad \cF_2 = \sigma(Y, Z), \quad \cF_3 = \sigma(X + Z, Y), \quad \cF_4 = \sigma(X + Y, X - Y), \quad \cF_5 = \sigma(X, Y, X + Y).
    \]

    \begin{enumerate}
        \item {[4]} For each sigma-algebra provide two examples of non-trivial (different from $\emptyset$ and $\Omega$) events that belong to it. 
        \item {[3]} Which of the sigma-algebras are always equal?
        \item {[3]} Which sigma-algebra is always a subset of another one?
    \end{enumerate}
    

    \item {[10]} The random variables $(X_k)$ are independent and uniform on $[0;2]$ and $Y = X_1 + 2X_2 + \dots + 5X_5 + 10$.
    \begin{enumerate}
        \item {[4]} Find the moment generating function of $X_1$.
        \item {[3]} Find the moment generating function of $Y$.
        \item {[3]} Find $\Var(Y \mid X_2)$.
        %\item {[4]} Find the moment generating function of $R_n = (S_n - n/2) / \sqrt{n/12}$.
    \end{enumerate}

  % под замену!
    \item {[10]} Gleb Zheglov catches one criminal every day. 
    With probability $0.2$ the catched criminal is replaced by $2$ new criminals. 
    Initially there is $1$ criminal in the town. 
    
    Let $T$ be the day of the ultimate crime eradication in the town. 
    
    \begin{enumerate}
      \item {[4]} Find $\E(T)$.
      \item {[6]} Find $\Var(T)$.
    \end{enumerate}

    \item {[10]} The random varible $(X_k)$ are independend and uniform on $[0;1]$.
    Let $Y_n = X_1 \cdot X_2 \cdot \dots \cdot X_k$.
    \begin{enumerate}
        \item {[5]} Does $(X_n)$ converge in probability? In distribution? Explain. 
        \item {[5]} Does $(Y_n)$ converge in probability? In distribution? Explain.
    \end{enumerate}
    
    \item {[10]} The random variables $(X_n)$ are independent and they have exponential distribution with rate $\lambda = 2$. 
    Consider the cumulative sum $S_n = X_1 + X_2 + \dots + X_n$ with $S_0 = 0$
    and the natural filtration $\cF_n = \sigma(X_1, X_2, \dots, X_n)$.

    \begin{enumerate}
        \item {[4]} Is $\cF = \cF_9 \backslash \cF_7$ a sigma-algebra? Why?
        \item {[6]} Find all constants $a$ and $b$ such that $M_n = S_n + a + b \cdot n$ is a martingale. 
    \end{enumerate}

    Hint: if $R \sim \dExpo(\lambda)$ then $\E(R) = 1/\lambda$ and $\Var(R) = 1/\lambda^2$. 

\end{enumerate}


\end{document}


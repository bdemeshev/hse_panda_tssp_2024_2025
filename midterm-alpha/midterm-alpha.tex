% arara: xelatex
\documentclass[12pt]{article}

% \usepackage{physics}

\usepackage{hyperref}
\hypersetup{
    colorlinks=true,
    linkcolor=blue,
    filecolor=magenta,      
    urlcolor=cyan,
    pdftitle={Overleaf Example},
    pdfpagemode=FullScreen,
    }

\usepackage{tikzducks}

\usepackage{tikz} % картинки в tikz
\usetikzlibrary{shapes, arrows, positioning}
\usepackage{microtype} % свешивание пунктуации

\usepackage{array} % для столбцов фиксированной ширины

\usepackage{indentfirst} % отступ в первом параграфе

\usepackage{sectsty} % для центрирования названий частей
\allsectionsfont{\centering}

\usepackage{amsmath, amsfonts, amssymb} % куча стандартных математических плюшек

\usepackage{comment}

\usepackage[top=2cm, left=1.2cm, right=1.2cm, bottom=2cm]{geometry} % размер текста на странице

\usepackage{lastpage} % чтобы узнать номер последней страницы

\usepackage{enumitem} % дополнительные плюшки для списков
%  например \begin{enumerate}[resume] позволяет продолжить нумерацию в новом списке
\usepackage{caption}

\usepackage{url} % to use \url{link to web}


\newcommand{\smallduck}{\begin{tikzpicture}[scale=0.3]
    \duck[
        cape=black,
        hat=black,
        mask=black
    ]
    \end{tikzpicture}}

\usepackage{fancyhdr} % весёлые колонтитулы
\pagestyle{fancy}
\lhead{Time Series and Stochastic Processes}
\chead{}
\rhead{Midterm exam}
\lfoot{}
\cfoot{}
\rfoot{}

\renewcommand{\headrulewidth}{0.4pt}
\renewcommand{\footrulewidth}{0.4pt}

\usepackage{tcolorbox} % рамочки!

\usepackage{todonotes} % для вставки в документ заметок о том, что осталось сделать
% \todo{Здесь надо коэффициенты исправить}
% \missingfigure{Здесь будет Последний день Помпеи}
% \listoftodos - печатает все поставленные \todo'шки


% более красивые таблицы
\usepackage{booktabs}
% заповеди из докупентации:
% 1. Не используйте вертикальные линни
% 2. Не используйте двойные линии
% 3. Единицы измерения - в шапку таблицы
% 4. Не сокращайте .1 вместо 0.1
% 5. Повторяющееся значение повторяйте, а не говорите "то же"


\setcounter{MaxMatrixCols}{20}
% by crazy default pmatrix supports only 10 cols :)


\usepackage{fontspec}
\usepackage{libertine}
\usepackage{polyglossia}

\setmainlanguage{russian}
\setotherlanguages{english}

% download "Linux Libertine" fonts:
% http://www.linuxlibertine.org/index.php?id=91&L=1
% \setmainfont{Linux Libertine O} % or Helvetica, Arial, Cambria
% why do we need \newfontfamily:
% http://tex.stackexchange.com/questions/91507/
% \newfontfamily{\cyrillicfonttt}{Linux Libertine O}

\AddEnumerateCounter{\asbuk}{\russian@alph}{щ} % для списков с русскими буквами
% \setlist[enumerate, 2]{label=\asbuk*),ref=\asbuk*}

%% эконометрические сокращения
\DeclareMathOperator{\Cov}{\mathbb{C}ov}
\DeclareMathOperator{\Corr}{\mathbb{C}orr}
\DeclareMathOperator{\Var}{\mathbb{V}ar}
\DeclareMathOperator{\col}{col}
\DeclareMathOperator{\row}{row}

\let\P\relax
\DeclareMathOperator{\P}{\mathbb{P}}

\DeclareMathOperator{\E}{\mathbb{E}}
% \DeclareMathOperator{\tr}{trace}
\DeclareMathOperator{\card}{card}

\DeclareMathOperator{\Convex}{Convex}
\DeclareMathOperator{\plim}{plim}

\newcommand{\cN}{\mathcal{N}}
\newcommand{\cF}{\mathcal{F}}

\newcommand{\RR}{\mathbb{R}}
\newcommand{\NN}{\mathbb{N}}
\newcommand{\hb}{\hat{\beta}}
\newcommand{\dPois}{\mathrm{Pois}}





\begin{document}

\begin{enumerate}
    \item {[10]} Michael stands in a corner of a hexagonal room with white soft walls. 
    Stochastic Processes course has caused him a deep trauma.
    Michael has insomnia and two independent personal identities.     
    At each iteration each personal identity moves from one vertex of the hexagon to the adjacent one.
    Two personal identities move independently with equal probabilities in both directions. 
    
    Consider the Markov chain where the state is a relative position of two identities of Michael in a hexagon. 
    \begin{enumerate}
        \item {[4]} Draw the diagram of chain states and find the transition matrix. 
        \item {[2]} Classify the states as transient or recurrent. 
        \item {[4]} Which proportion of his eternal life Michael spends in a perfect harmony with himself (two personal identities are located at the same vertex)?
    \end{enumerate}


    \item {[10]} Compare sigma-algebras and conditional expected values!
    
    \begin{enumerate}
        \item {[5]} Consider sigma-algebras 
        \[
        \cF_1 = \sigma(X), \quad \cF_2 = \sigma(Y), \quad \cF_3 = \sigma(X, Y), \quad \cF_4 = \sigma(X + Y, X - Y). %, \quad \cF_5 = \sigma(X, Y, X + Y).
        \]
        Which of them are always equal? Which sigma-algebra is always a subset of another one?

        \item {[5]} Consider random variables 
        \[
        R_1 = \E(X \mid Y), \quad R_2 = \E(1 / X \mid Y), \quad R_3 = \E(1 / X \mid 1/Y), \quad R_4 = \E(X \mid 1/Y). %, R_5 = 1 / \E(X \mid Y).
        \]
        Which of them are always equal provided that they exist?
    \end{enumerate}
    
    

    \item {[10]} The random variables $(X_n)$ are independent and uniform on $[0;1]$ and $S_n = X_1 + X_2 + \dots + X_n$.
    \begin{enumerate}
        \item {[4]} Find the moment generating function of $X_1$.
        \item {[2]} Find the moment generating function of $R = S_{10} - 5$.
        \item {[4]} Find $\E(X_1 \mid S_3)$.
        %\item {[4]} Find the moment generating function of $R_n = (S_n - n/2) / \sqrt{n/12}$.
    \end{enumerate}

    \newpage

    \item {[10]} I throw a fair coin. 
    Let $Y$ be the number of throws until I obtain the sequence head-tail-head. 

    \begin{enumerate}
        \item {[4]} Find $\E(Y)$.
        \item {[6]} Find $\E(Y^2)$ and hence $\Var(Y)$.
    \end{enumerate}


    \item {[10]} Consider two non-random sequences, $h_n = 1/n$ and $t_n = 1 + 1/n$.
    Elon Musk throws a fair coin once and selects the sequence $(h_n)$ if it lands on head and selects the sequence $(t_n)$ otherwise.
    Hence Elon obtains a random sequence $(X_n)$.

    \begin{enumerate}
        \item {[2]} What is the distribution of $X_5$?
        \item {[4]} What is the distribution of $\lim X_n$?
        \item {[4]} Write $\plim X_n$ explicitely as a function of $X_1$.
    \end{enumerate}
    
    \item {[10]} The random variables $(X_n)$ are independent and they have Poisson distribution with rate $\lambda = 2$. 
    Consider the cumulative sum $S_n = X_1 + X_2 + \dots + X_n$ with $S_0 = 0$
    and the natural filtration $\cF_n = \sigma(X_1, X_2, \dots, X_n)$.

    \begin{enumerate}
        \item {[4]} Provide two examples of events that belong to $\cF_9$ but do not belong to $\cF_7$.
        \item {[6]} Find all constants $a$ and $b$ such that $M_n = S_n + a + b \cdot n$ is a martingale. 
    \end{enumerate}

    Hint: if $R \sim \dPois(\lambda)$ then $\E(R) = \lambda$ and $\Var(R) = \lambda$. 

\end{enumerate}


\end{document}


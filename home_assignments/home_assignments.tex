% arara: xelatex
\documentclass[12pt]{article}

% Stochastic Processes, 2024-2025

% \usepackage{physics}

\usepackage{hyperref}
\hypersetup{
    colorlinks=true,
    linkcolor=blue,
    filecolor=magenta,      
    urlcolor=cyan,
    pdftitle={Overleaf Example},
    pdfpagemode=FullScreen,
    }

\usepackage{tikzducks}

\usepackage{tikz} % картинки в tikz
\usetikzlibrary{shapes, arrows, positioning}
\usepackage{microtype} % свешивание пунктуации

\usepackage{array} % для столбцов фиксированной ширины

\usepackage{indentfirst} % отступ в первом параграфе

\usepackage{sectsty} % для центрирования названий частей
\allsectionsfont{\centering}

\usepackage{amsmath, amsfonts, amssymb} % куча стандартных математических плюшек

\usepackage{comment}

\usepackage[top=2cm, left=1.2cm, right=1.2cm, bottom=2cm]{geometry} % размер текста на странице

\usepackage{lastpage} % чтобы узнать номер последней страницы

\usepackage{enumitem} % дополнительные плюшки для списков
%  например \begin{enumerate}[resume] позволяет продолжить нумерацию в новом списке
\usepackage{caption}

\usepackage{url} % to use \url{link to web}


\newcommand{\smallduck}{\begin{tikzpicture}[scale=0.3]
    \duck[
        cape=black,
        hat=black,
        mask=black
    ]
    \end{tikzpicture}}

\usepackage{fancyhdr} % весёлые колонтитулы
\pagestyle{fancy}
\lhead{Time Series and Stochastic Processes}
\chead{}
\rhead{Home assignments for samurai}
\lfoot{}
\cfoot{}
\rfoot{}

\renewcommand{\headrulewidth}{0.4pt}
\renewcommand{\footrulewidth}{0.4pt}

\usepackage{tcolorbox} % рамочки!

\usepackage{todonotes} % для вставки в документ заметок о том, что осталось сделать
% \todo{Здесь надо коэффициенты исправить}
% \missingfigure{Здесь будет Последний день Помпеи}
% \listoftodos - печатает все поставленные \todo'шки


% более красивые таблицы
\usepackage{booktabs}
% заповеди из докупентации:
% 1. Не используйте вертикальные линни
% 2. Не используйте двойные линии
% 3. Единицы измерения - в шапку таблицы
% 4. Не сокращайте .1 вместо 0.1
% 5. Повторяющееся значение повторяйте, а не говорите "то же"


\setcounter{MaxMatrixCols}{20}
% by crazy default pmatrix supports only 10 cols :)


\usepackage{fontspec}
\usepackage{libertine}
\usepackage{polyglossia}

\setmainlanguage{russian}
\setotherlanguages{english}

% download "Linux Libertine" fonts:
% http://www.linuxlibertine.org/index.php?id=91&L=1
% \setmainfont{Linux Libertine O} % or Helvetica, Arial, Cambria
% why do we need \newfontfamily:
% http://tex.stackexchange.com/questions/91507/
% \newfontfamily{\cyrillicfonttt}{Linux Libertine O}

\AddEnumerateCounter{\asbuk}{\russian@alph}{щ} % для списков с русскими буквами
% \setlist[enumerate, 2]{label=\asbuk*),ref=\asbuk*}

%% эконометрические сокращения
\DeclareMathOperator{\Cov}{\mathbb{C}ov}
\DeclareMathOperator{\Corr}{\mathbb{C}orr}
\DeclareMathOperator{\Var}{\mathbb{V}ar}
\DeclareMathOperator{\col}{col}
\DeclareMathOperator{\row}{row}

\let\P\relax
\DeclareMathOperator{\P}{\mathbb{P}}

\DeclareMathOperator{\E}{\mathbb{E}}
% \DeclareMathOperator{\tr}{trace}
\DeclareMathOperator{\card}{card}

\DeclareMathOperator{\Convex}{Convex}
\DeclareMathOperator{\plim}{plim}

\newcommand{\cF}{\mathcal{F}}
\newcommand{\cH}{\mathcal{H}}



\newcommand{\cN}{\mathcal{N}}
\newcommand{\RR}{\mathbb{R}}
\newcommand{\NN}{\mathbb{N}}
\newcommand{\hb}{\hat{\beta}}





\begin{document}

Be brave! You can use python. In this case just provide the code. 
You can use ChatGPT or any other LLM. In this case just provide the full prompt. 
Don't panic!


\section*{Home assignment 1}


Deadline: 2024-09-23, 21:00.

\begin{enumerate}
\item The Cat can be only in two states: Sleeping ($S$) and Eating ($E$). 
Cat's mood depends only on the previous state. 
The transition probabilities are given below:

%\begin{minipage}
\begin{tikzpicture}[->,>= stealth', shorten >=2pt , line width=0.5pt, node distance=2cm]
    \node [circle, draw] (one) {$E$};
    \node [circle, draw] (two) [right of=one] {$S$};
    \path (one) edge [bend left] node [above] {$0.1$} (two);
    \path (two) edge [bend left] node [below]{$0.2$} (one);
    \path (two) edge [loop right] node {} (two);
    \path (one) edge [loop left] node {} (one);
\end{tikzpicture}
%\end{minipage}


\begin{enumerate}
    \item Compute the missing probabilities on the graph.
    \item Write down the transition matrix.
    \item Compute $\P(X_3 = \text{Eating} \mid X_0 = \text{Eating})$.
\end{enumerate}

\item Cowboy Joe enters the Epsilon Bar and orders one pint of beer. 
He drinks it and orders one pint more. 
And so on and so on and so on\dots{ }
The problem is that the barmaid waters down each pint with probability $0.2$ independently of other pints.
Joe does not like watered down beer. 
He will blow the Epsilon Bar to hell if two or more out of the last three pints are watered down. 

We point out that Joe never drinks less than 3 pints in a bar. 

\begin{enumerate}
    \item What is the expected number of pints of beer Joe will drink?
\end{enumerate}

Let $Y_t$ be the indicator that the pint number $t$ was watered down. 
Consider the Markov chain $S_t = (y_{t-2}, y_{t-1}, y_t)$.
For example, $S_t = (100)$ means that the pint number $t-2$ was watered down while pints number $t-1$ and $t$ are good. 

\begin{enumerate}[resume]
\item What are the possible values of $S_3$ and their probabilities?
\item Write down the transition matrix of this Markov chain.
\end{enumerate}

Note: questions (2b) and (2c) were updated!

\item Pavel Durov starts at the point $X_0 = 3$ on the real line. 
Each minute he moves left with probability $0.4$ or right with probability $0.6$ independently of past moves.
The points $0$ and $5$ are absorbing. 
If Pavel reaches $0$ or $5$ he stays there forever.
Let $X_t$ be the coordinate of Pavel after $t$ minutes. 

\begin{enumerate}
    \item Write down the transition matrix of this Markov chain. 
    \item Calculate the distribution of $X_7$ [list all values of the random variable $X_7$ and estimate the probabilities].
\end{enumerate}



Hint: you are free to use python for this problem :)


\end{enumerate}


\section*{Home assignment 2}

Deadline: 2024-09-27, 21:00.

\begin{enumerate}
    \item {[10 points]} Consider two identical hedgehogs starting at the vertices $A$ and $B$ of a polygon $ABCD$. 
    Each minute each hedehog simulteneously and independently chooses to go clockwise to the adjacent point, to go counter-clockwise to the adjacent point or to stay at his location.
    
    Thus the brotherhood of two brave hedgehogs can be in three states: in one vertex, 
    in two adjacent vertices, in two non-adjacent vertices.

  
    \begin{enumerate}
      \item Draw the graph for the brotherhood Markov chain and calculate all transition probabilities. 
      \item Write down the transition matrix of the brotherhood Markov chain. 
      \item What is the probability that they will be in one vertex after 3 steps?
    \end{enumerate}
    
    \item {[10 points]} Consider the following Markov chain:

    \begin{tikzpicture}[->,>= stealth', shorten >=2pt , line width=0.5pt, node distance=2cm]
        \node [circle, draw] (one) {$A$};
        \node [circle, draw] (two) [right of=one] {$B$};
        \node [circle,  draw] (three) [right of=two] {$C$};
        \path (one) edge [bend left] node [above] {$0.3$} (two);
        \path (two) edge [bend right] node [below] {$0.3$} (three);
        \path (two) edge [bend left] node [below]{$0.7$} (one);
        \path (three) edge [loop right] node {$0.2$} (three);
        \path (three) edge [bend right] node [above] {$0.8$} (two);
        \path (one) edge [loop left] node {$0.7$} (one);
    \end{tikzpicture}

    \begin{enumerate}
        \item Find the stationary distribution of this Markov chain.
    \end{enumerate}
    
    The Markov chains starts at the vertex $A$.
    Let $N$ be the first moment when the state $C$ will be reached.

    \begin{enumerate}[resume]
      \item Find the expected value $\E(N)$.
      \item Find the variance $\Var(N)$. 
    \end{enumerate}

    \item {[10 points]} Bonnie and Clyde start at the points $(5, 0)$ and $(-5, 0)$ of the plane. 
    Each minute each of them simulteneously and independently makes one step in one of the four possible directions (south, north, east, west).

    Each of them does $n$ steps.
    Let $X$ be the number of times they will be at the same point.
    \begin{enumerate}
        \item Estimate the probability $\P(X \geq 1)$ for $n=50$ using $B=10000$ simulations. 
        \item Estimate $\E(X)$ and $\Var(X)$ for $n=50$ using $B=10000$ simulations. 
        \item Plot the estimated value of $\E(X)$ as a function of $n$ for $n$ from $1$ to $200$ using $B=10000$ simulations. 
    \end{enumerate}

\end{enumerate}




\section*{Home assignment 3}

Deadline: 2024-10-04, 23:59.

\begin{enumerate}
    \item {[10 points]} We randomly wander on the graph choosing at each moment of time one of the possible directions equiprobably.
    
    \begin{tikzpicture}[->,>= stealth', shorten >=2pt , line width=0.5pt, node distance=2cm]
        \node [circle, draw] (A) {$a$};
        \node [circle, draw] (B) [right of=A] {$b$};
        \node [circle, draw] (C) [right of=B] {$c$};
        \node [circle, draw] (D) [below of=A] {$d$};
        \node [circle, draw] (E) [right of=D] {$e$};
        \node [circle, draw] (F) [right of=E] {$f$};
        \node [circle, draw] (G) [right of=C] {$g$};
      
      
        \path (A) edge (D);
        \path (D) edge (E);
        \path (E) edge (A);
        \path (B) edge (E);
        \path (B) edge (C);
        \path (C) edge (F);
        \path (C) edge (G);
        \path (F) edge (B);
      
      
        \path (D) edge [loop left] (D);
        \path (E) edge [loop right] (E);
        \path (G) edge [loop below] (G);
    \end{tikzpicture}
    \begin{enumerate}
        \item Split each Markov chain into communicating classes. 
        \item Find the period of every state. 
        \item Classify each state as transient or recurrent.
        \item For recurrent states find the expected return time.
        \item Find the stationary distributions. 
    \end{enumerate}
        
    
    
    \item {[10 points]} Design a Markov chain with 3 states and unique stationary distribution $\pi = (0.1, 0.2, 0.7)$.

    \item {[10 points]} Consider three games:
    \begin{enumerate}
        \item[Game A:] You toss a biased coin with probability $0.48$ of $H$. 
        You get $+1$ dollar for $H$ and $-1$ dollar for $T$.  

        \item[Game B:] If your wellfare is divisible by three you toss a coin that
        lands on $H$ with probability $0.09$. 
        If your wellfare is not divisible by three you toss a coin that lands on $H$ with probability $0.74$.
        You get $+1$ dollar for $H$ and $-1$ dollar for $T$.  
        
        \item[Game C:] You toss an unbiased coin.
        If it lands on $H$ you play Game A. If it lands on $T$ you play Game B. 
    \end{enumerate}
    
    Your initial capital is $10000\$$.

    \begin{enumerate}
        \item Generate and plot two random trajectories of your wellfare if you play Game A $10^6$ times.
        \item Generate and plot two random trajectories of your wellfare if you play Game B $10^6$ times.
        \item Generate and plot two random trajectories of your wellfare if you play Game C $10^6$ times.
    \end{enumerate}


\end{enumerate}


\section*{Home assignment 4}

Deadline: 2024-10-14, 23:59.

\begin{enumerate}
    \item Recognise the distribution family and its parameters by looking at the moment-generating function:
    
    \begin{enumerate}
        \item $0.7 + 0.3\exp(t)$;
        \item $\exp(2024\exp(t)) / \exp(2024)$;
        % \item $\frac{\exp(3t) - 1}{3t\exp(-2t)}$;
        \item $\exp(6t + 2024t^2)$;
        \item $1/(5t - 1)^{2024}$.
    \end{enumerate}
    
    You may use the table from the article 
    
    \url{https://en.wikipedia.org/wiki/Moment-generating_function}.
    
    \item Consider the moment-generating function of a random variable $X$:
    \[
     g(t) = \frac{\exp(3t) - 1}{3t\exp(-2t)}.
    \]
    
    \begin{enumerate}
        \item Expand the function $g(t)$ as Taylor series up to $t^4$ included. 
        \item Find $\E(X)$, $\E(X^2)$, $\E(X^3)$, $\E(X^4)$.
    \end{enumerate}


    \item The moment-generating function of the pair of random variables $(X, Y)$ is given by 
    $\exp(6t_1 + 5t_2 + t_1^2 + 20t_2^2 - 2t_1t_2)$.

    Find $\E(X)$, $\Var(Y)$, $\E(XY)$.
    
\end{enumerate}



\section*{Home assignment 5}

Deadline: 2024-10-18, 23:59.

\begin{enumerate}

\item {[10 points]} The random variables $X_i$ are independent and exponentially distributed with rate $\lambda = 1$. 

\begin{enumerate}
    \item Find the probability limit
    \[
    \plim \frac{X_1 + X_2 + X_3 + \dots + X_n}{2n + 7}.
    \]
    \item Find the probability limit
    \[
    \plim \frac{X_1^2 + X_2^2 + X_3^2 + \dots + X_n^2}{2n + 7}.
    \]
    \item Find the probability limit
    \[
        \plim \min\{X_1, X_2, X_3, \dots, X_n\}.
    \]
    \item Find the probability limit
    \[
    \plim \sqrt[n]{\exp(2X_1 + 2X_2 + \dots + 2X_n)}.
    \]
\end{enumerate}



\item {[10 points]} Polina loves sweet chestnuts. 
She has infinite sequence of baskets before her.
In the basket number~$n$ there are $n$~chestnuts in total.
Unfortunately only one chestnut in every basket is a sweet one. 

She picks chestnuts one by one at random from all the buskets sequentially. 
First she picks the unique chestnut from the basket number one, 
than she picks in a random order two chestnuts from the basket number two and so on. 

The random variable $S_t$ indicates whether the chestnut number $t$ was a sweet one. 

\begin{enumerate}
  \item Find $\lim S_t$ or prove that the limit does not exist.
  \item Find $\plim S_t$ or prove that the limit does not exist.
  \item Find mean square limit of $S_t$ or prove that the limit does not exist. 
\end{enumerate}

\item {[10 points]} Consider two Markov chains, $(X_t)$ and $(Y_t)$: 

\begin{tikzpicture}[->,>= stealth', shorten >=2pt , line width=0.5pt, node distance=3cm]
    \node [circle, draw] (one) {$x = -1$};
    \node [circle, draw] (two) [right of=one] {$x = 0$};
    \node [circle,  draw] (three) [right of=two] {$x = 1$};
    % \path (one) edge [bend left] node [above] {$1$} (two);
    \path (two) edge node [below] {$0.3$} (three);
    \path (two) edge node [below]{$0.7$} (one);
    %\path (three) edge [loop right] node {$0.2$} (three);
    %\path (three) edge [bend right] node [above] {$0.8$} (two);
    \path (one) edge [loop left] node {$1$} (one);
    \path (three) edge [loop right] node {$1$} (three);
\end{tikzpicture} with $X_0 = 0$;

and

\begin{tikzpicture}[->,>= stealth', shorten >=2pt , line width=0.5pt, node distance=3cm]
    \node [circle, draw] (one) {$y = -1$};
    \node [circle, draw] (two) [right of=one] {$y = 0$};
    \node [circle,  draw] (three) [right of=two] {$y = 1$};
    \node [circle,  draw] (four) [right of=three] {$y = 2$};
    % \path (one) edge [bend left] node [above] {$1$} (two);
    \path (two) edge node [below] {$0.3$} (three);
    \path (two) edge node [below]{$0.7$} (one);
    % \path (three) edge [loop right] node {$0.2$} (three);
    \path (three) edge [bend left] node [above] {$0.5$} (four);
    \path (four) edge [bend left] node [above] {$1$} (three);
    \path (one) edge [loop left] node {$1$} (one);
    \path (three) edge [loop below] node {$0.5$} (three);
\end{tikzpicture} with $Y_0 = 0$.



\begin{enumerate}
    \item Find $\P(\lim X_n \text{ exists})$ and $\P(\lim Y_n \text{ exists})$.
    \item Find the limiting distribution of $(X_n)$ and the limiting distribution of $(Y_n)$.

    Hint: here you need to calculate all limits $\lim \P(X_n = k)$, $\lim \P(Y_n = k)$.
    
    \item Does $(X_n)$ converges almost surely? In distribution? In probability?
    \item Does $(Y_n)$ converges almost surely? In distribution? In probability?
\end{enumerate}


\end{enumerate}


\section*{Home assignment 6}

Deadline: 2024-10-26, 23:59.

\begin{enumerate}

\item {[10 points]} Albert Nikolayevich Shiryaev randomly selects a natural number $N$ from $1$ to $7$.
Let $Y$ be the remainder after division of $N$ by $2$ and $X$ be the remainder after division of $N$ by $3$. 

\begin{enumerate}
    \item Write the joint probability table for $(X, Y)$.
    \item Find $\E(Y \mid X)$. Is it linear in $X$?
    % \item Find $\E(X \mid Y)$. Is it linear in $Y$?
    \item Find $\E(\E(Y \mid X))$ and $\Var(\E(Y \mid X))$.
    \item Find $\Var(Y \mid X)$.
    \item Find $\E(\Var(Y \mid X))$.
    % \item Find $\E(\Var(Y \mid X)) + \Var(\E(Y \mid X))$.
\end{enumerate}


\item {[10 points]} Albert Nikolayevich selects a random point uniformly inside a quadrilateral $ABCD$ where $A = (0, 0)$, $B = (0, 2)$,
$C = (4, 4)$, $D = (4, 0)$.

\begin{enumerate}
    \item Find $\E(Y \mid X)$ and $\E(X \mid Y)$.
    \item Find $\Var(Y \mid X)$ and $\Var(X \mid Y)$.
\end{enumerate}

Hint: you may use the formula for the variance of uniform distribution :)

\item {[10 points]} Albert Nikolayevich selects a random point $(X, Y)$ with joint probability density
\[
f(x, y) = \begin{cases}
    (3x^2 + 4y^3) / 2, \text{ if } x \in [0;1], y \in [0;1]; \\
    0, \text{ otherwise}. 
\end{cases}
\]
\begin{enumerate}
    \item For the random variable $x$ find the marginal probability density function $f(x)$.
    %\footnote{%
    %Hereinafter we abuse notation. Strictly speaking we shoud write $f_X(x)$ and $f_Y(y)$ because these are different functions.} 
    \item Find the conditional density $f(y \mid x)$.
    \item Find the conditional expected value $\E(Y \mid X)$. Is it linear in $X$?
    \item Find $\Var(Y \mid X)$. Is it constant?
    \item Find $\E(X)$, $\E(Y)$, $\Cov(X, Y)$ and $\Var(X)$.
    % \item Find the best linear approximation $Y^* = \beta_0 + \beta_1 X$ of the random variable $Y$.
    
    % Hint: Here you should minimize $\E((Y - Y^*)^2)$ with respect to the true constants $\beta_0$ and $\beta_1$.
\end{enumerate}

\end{enumerate}



\section*{Home assignment 7}

Deadline: 2024-11-03, 23:59.

\begin{enumerate}
\item Experiment may end by one of the six outcomes:

\begin{tabular}{*{4}{c}}
\toprule
& $X=-2$ & $X=0$ & $X=2$ \\
\midrule
$Y=-1$ & 0.1 & 0.2 & 0.3  \\
$Y=1$ & 0.2 & 0.1 & 0.1  \\
\bottomrule
\end{tabular}

\begin{enumerate}
  \item Find expicitely the sigma-algebras $\sigma(X)$, $\sigma(Y)$, $\sigma(X \cdot Y)$.
  \item How many elements are there in $\sigma(X + Y)$, $\sigma(X - Y)$?
  \item Calculate conditional expected values $\E(X \mid \sigma(Y))$, $\E(X \mid \sigma(X + Y))$.
\end{enumerate}


\item We throw a coin infinitely many times. 
Let $X_{n}$ be the indicator that the coin landed on Head at toss number $n$.
Consider a pack of $\sigma$-algebras: $\cF_{n}:=\sigma(X_1, X_2, \ldots, X_n)$, $\cH_{n}:=\sigma(X_{n}, X_{n+1}, X_{n+2}, \ldots)$.

\begin{enumerate}
\item Simplify experessions: $\cF_{11}\cap \cF_{25}$, $\cF_{11}\cup \cF_{25}$, $\cH_{11}\cup \cH_{25}$.

\item For each case provide two examples of $\sigma$-algebras that contain the corresponding event
\begin{enumerate}
\item $\{X_{37}>0 \}$;
\item $\{X_{37}>X_{2024} \}$;
\item $\{ X_{37}>X_{2024}>X_{12} \}$;
\end{enumerate}

\item For each case provide two non-trivial examples (different from $\Omega$ and $\emptyset$) of an event $A$ such that

\begin{enumerate}
\item $A\in \cF_{2024}$;
\item $A\notin \cF_{2025}$;
\item $A \in \cH_{n}$ for all possible $n$;
\end{enumerate}


\end{enumerate}

\item Consider a fair dice. 
In the experiment we throw the dice until the first six appears.

\begin{enumerate}
    \item Simulate $B = 100000$ experiments. 
    For every experiment number $i$ record the total number of throws, $y_i$, and the number of even faces appeared, $x_i$.
    \item For all values of $x$ where you have more than $100$ records estimate $\hat\mu(x) = \hat{\E}(y_i \mid x_i = x)$ and $\hat v(x) = \widehat{\Var}(y_i \mid x_i = x)$.
    \item Explain intuitively why $\hat\mu(0)$ is less than $3$. 
    \item Randomly select $100$ experiments out of all $B$ experiments.
    Draw the scatter plot $(x_i, y_i)$ for randomly selected experiments.
    Add the line $\hat\mu(x)$ with bands $\hat\mu(x) \pm 2\sqrt{\hat v(x)}$ to the scatter plot. 
    \item Is it reasonable to assume that $\hat \mu(x)$ is linear?
    \item Is it reasonable to assume that $\hat v(x)$ is constant?
\end{enumerate}

No formal tests are required for the last two questions, graphical analysis is sufficient.


\end{enumerate}

\end{document}









\end{document}

